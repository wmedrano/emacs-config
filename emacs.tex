% Created 2025-05-26 Mon 18:20
% Intended LaTeX compiler: pdflatex
\documentclass[11pt]{article}
\usepackage[utf8]{inputenc}
\usepackage[T1]{fontenc}
\usepackage{graphicx}
\usepackage{longtable}
\usepackage{wrapfig}
\usepackage{rotating}
\usepackage[normalem]{ulem}
\usepackage{amsmath}
\usepackage{amssymb}
\usepackage{capt-of}
\usepackage{hyperref}
\author{will@wmedrano.dev}
\date{\today}
\title{Emacs Config}
\hypersetup{
 pdfauthor={will@wmedrano.dev},
 pdftitle={Emacs Config},
 pdfkeywords={},
 pdfsubject={},
 pdfcreator={Emacs 30.1 (Org mode 9.7.11)}, 
 pdflang={English}}
\begin{document}

\maketitle
\tableofcontents

\section{Packages}
\label{sec:org9bf85c0}

\subsection{Melpa}
\label{sec:org7915c61}

Add Melpa to the package manager. \href{https://melpa.org}{Melpa} contains many popular Emacs packages.

\begin{verbatim}
(require 'package)
(add-to-list 'package-archives '("melpa" . "https://melpa.org/packages/") t)
\end{verbatim}

Counting the default package archives, the following package archives are
available.

\begin{verbatim}
package-archives
\end{verbatim}

\phantomsection
\label{orgf010245}
\begin{verbatim}
((gnu . https://elpa.gnu.org/packages/) (nongnu . https://elpa.nongnu.org/nongnu/) (melpa . https://melpa.org/packages/))
\end{verbatim}


Package archives must be manually refreshed or fetched with \texttt{M-x
package-refresh-contents}. All packages added to the \texttt{package-selected-packages}
variable can be installed with \texttt{M-x package-install-selected-packages}. Packages
can also be installed on a one-off basis interactively with \texttt{M-x
package-install}.

In some cases, package installation may fail with "\ldots{} not found". This likely
means that the package archives point to an old (and non-existant) version of
the package. The package definitions can be updated by running \texttt{M-x
package-refresh-contents}. Packages may also be upgraded all at once with \texttt{M-x
package-upgrade-all}.
\subsection{Initialize}
\label{sec:org7bc0fbc}

Initialize the package archive. This makes all previously installed packages available.

\begin{verbatim}
(package-initialize)
\end{verbatim}
\section{Startup}
\label{sec:orgd737570}

Disable the default Emacs startup screen. Instead, this displays just the opened
file or the \texttt{*Scratch*} buffer if no file has been opened.

\begin{verbatim}
(setq-default inhibit-startup-screen t)
\end{verbatim}
\section{Backups \& Autosaves}
\label{sec:org64d7226}

Backup and autosaves may litter the filesystem so we disable
them. This is ok as my disk is reliable, I save often, and use version
control.

\begin{verbatim}
(setq-default auto-save-interval 0
              create-lockfiles   nil
              make-backup-files  nil)
\end{verbatim}
\section{Appearance \& Feel}
\label{sec:org2a5bb84}

\subsection{Remove Clutter}
\label{sec:orge83e8ce}

Remove the menu bar and tool bar.

\begin{verbatim}
(menu-bar-mode -1)
(tool-bar-mode -1)
\end{verbatim}

Disable the scroll bar. The functionality is ok sometimes, but it clashes with
the theming.

\begin{verbatim}
(scroll-bar-mode -1)
\end{verbatim}
\subsection{Lines}
\label{sec:orgb0e77f2}

Scroll conservatively values above 100 cause Emacs to scroll the
minimum number of lines required to get the cursor in position. The
default value of 0 causes Emacs to recenter the window.

\begin{verbatim}
(setq-default scroll-conservatively 101)
\end{verbatim}

Display line numbers for text buffers. This can be toggled in an individual
buffer with \texttt{M-x display-line-numbers-mode}.

\begin{verbatim}
(global-display-line-numbers-mode t)
\end{verbatim}

Highlight the currently selected line. This can be toggled in an individual
buffer with \texttt{M-x hl-line-mode}.

\begin{verbatim}
(global-hl-line-mode t)
\end{verbatim}
\subsection{Color Scheme}
\label{sec:org27b1498}

Use the \texttt{doom-dracula} theme from the \href{https://github.com/doomemacs/themes/tree/729ad034631cba41602ad9191275ece472c21941}{Doom Themes} package.

\begin{verbatim}
(add-to-list 'package-selected-packages 'doom-themes)
(load-theme 'doom-dracula t)
\end{verbatim}
\subsection{Modeline}
\label{sec:orgea572eb}

Use \href{https://github.com/seagle0128/doom-modeline/tree/297b57585fe3b3de9e694512170c44c6e104808f}{Doom Modeline} to display a nicer modeline. Mainly, it:

\begin{itemize}
\item Uses more icons.
\item Displays a minimal amount of information while still keeping
important information such as:
\begin{itemize}
\item Syntax errors
\item Version control information
\end{itemize}
\end{itemize}

\begin{verbatim}
(add-to-list 'package-selected-packages 'doom-themes)
(doom-modeline-mode t)
\end{verbatim}
\section{Editor Completions}
\label{sec:org7a05584}

Editor completions refers to auto complete done within the editor context, as
opposed to code. Editor completion is used to complete prompts for things such
as selecting a file, buffer, or command.
\subsection{Ivy}
\label{sec:org4b59c39}

Editor completions are displayed using the \href{https://github.com/abo-abo/swiper?tab=readme-ov-file\#ivy}{Ivy} package. This provides a huge
improvement over the default built-in Emacs completion.

\begin{verbatim}
(add-to-list 'package-selected-packages 'ivy)
(ivy-mode t)
\end{verbatim}
\subsection{Counsel}
\label{sec:org8decf39}

\href{https://github.com/abo-abo/swiper?tab=readme-ov-file\#counsel}{Counsel} provides functions that wrap ivy completion with some extra
features. For example, \texttt{counsel-M-x} is an \texttt{M-x} replacement that also displays
a keybinding if there is an active keybinding for the particular function.

\begin{verbatim}
(add-to-list 'package-selected-packages 'counsel)
(counsel-mode t)
\end{verbatim}

Enabling \texttt{counsel-mode} makes the \texttt{counsel-mode-map} keymap active. This keymap
defines several rebinds.

\begin{verbatim}
counsel-mode-map
\end{verbatim}

\begin{table}[htbp]
\label{tab:orge042f57}
\centering
\begin{tabular}{lll}
keymap & (24 keymap (98 . counsel-switch-buffer)) & (remap keymap (bookmark-jump . counsel-bookmark) (geiser-doc-look-up-manual . counsel-geiser-doc-look-up-manual) (pop-to-mark-command . counsel-mark-ring) (info-lookup-symbol . counsel-info-lookup-symbol) (yank-pop . counsel-yank-pop) (load-theme . counsel-load-theme) (load-library . counsel-load-library) (imenu . counsel-imenu) (find-library . counsel-find-library) (find-file . counsel-find-file) (list-faces-display . counsel-faces) (describe-face . counsel-describe-face) (apropos-command . counsel-apropos) (describe-symbol . counsel-describe-symbol) (describe-variable . counsel-describe-variable) (describe-function . counsel-describe-function) (describe-bindings . counsel-descbinds) (execute-extended-command . counsel-M-x))\\
\end{tabular}
\end{table}

However, it does not provide a rebind for \texttt{counsel-switch-buffer}. We make this
our default (interactive) switch buffer command as it allows previewing the
contents of a buffer before switching.

\begin{verbatim}
(define-key counsel-mode-map (kbd "C-x b") #'counsel-switch-buffer)
\end{verbatim}
\section{Formatting}
\label{sec:orgfaa509f}

\subsection{Tabs}
\label{sec:org7d0b79b}

Emacs uses a combination of tabs and spaces when auto-indenting. This pleases
neither the spaces crowd, nor the tabs crowd. Tabs are disabled to prevent the
mixed use, though opinionated languages will still find a way to use their
correct default. For example, Go will still use tabs when indenting.

\begin{verbatim}
(setq-default indent-tabs-mode nil)
\end{verbatim}

Use a default tab width of 4 spaces.

\begin{verbatim}
(setq-default tab-width 4)
\end{verbatim}
\subsection{Line Width}
\label{sec:org4e8035d}

Set a target line width of 80. Contents of a "paragraph" may be made to follow
the target line width through \texttt{M-x fill-paragraph} or a highlighted region with
\texttt{M-x fill-region}.

\begin{verbatim}
(setq-default fill-column 80)
\end{verbatim}

Some languages have a different target line length.

\begin{verbatim}
(defun fill-column-100 ()
  (setq-local fill-column 100))

(add-hook 'rust-mode-hook #'fill-column-100)
\end{verbatim}
\section{Languages}
\label{sec:org5a8c681}

\subsection{Rust}
\label{sec:org16dbc73}

\begin{verbatim}
(add-to-list 'package-selected-packages 'rust-mode)
\end{verbatim}
\subsection{Org Mode}
\label{sec:org50c7c89}

Enable syntax highlighting for exported material.

\begin{verbatim}
(add-to-list 'package-selected-packages 'htmlize)
\end{verbatim}

Enable previews while editing org document. Previews can be enabled with \texttt{M-x
org-preview-html-mode}. Behind the scenes, this exports to HTML on save on
displays the generated HTML in an \texttt{*eww*} buffer.

\begin{verbatim}
(add-to-list 'package-selected-packages 'org-preview-html)
\end{verbatim}
\end{document}
