% Created 2025-05-26 Mon 22:13
% Intended LaTeX compiler: pdflatex
\documentclass[11pt]{article}
\usepackage[utf8]{inputenc}
\usepackage[T1]{fontenc}
\usepackage{graphicx}
\usepackage{longtable}
\usepackage{wrapfig}
\usepackage{rotating}
\usepackage[normalem]{ulem}
\usepackage{amsmath}
\usepackage{amssymb}
\usepackage{capt-of}
\usepackage{hyperref}
\author{will@wmedrano.dev}
\date{\today}
\title{Emacs Config}
\hypersetup{
 pdfauthor={will@wmedrano.dev},
 pdftitle={Emacs Config},
 pdfkeywords={},
 pdfsubject={},
 pdfcreator={Emacs 30.1 (Org mode 9.7.11)}, 
 pdflang={English}}
\begin{document}

\maketitle
\tableofcontents

\section{Packages}
\label{sec:orgf41818d}

\subsection{Melpa}
\label{sec:org1ce1463}

Add Melpa to the package manager. \href{https://melpa.org}{Melpa} contains many popular Emacs packages.

\begin{verbatim}
(require 'package)
(add-to-list 'package-archives '("melpa" . "https://melpa.org/packages/") t)
\end{verbatim}

Counting the default package archives, the following package archives are
available.

\begin{verbatim}
(cl-loop for package-archive in package-archives
         collect (list (car package-archive) (cdr package-archive)))
\end{verbatim}

\begin{table}[htbp]
\label{tab:org09f32cc}
\centering
\begin{tabular}{ll}
gnu & \url{https://elpa.gnu.org/packages/}\\
nongnu & \url{https://elpa.nongnu.org/nongnu/}\\
melpa & \url{https://melpa.org/packages/}\\
\end{tabular}
\end{table}

Package archives must be manually refreshed or fetched with \texttt{M-x
package-refresh-contents}. All packages added to the \texttt{package-selected-packages}
variable can be installed with \texttt{M-x package-install-selected-packages}. Packages
can also be installed on a one-off basis interactively with \texttt{M-x
package-install}.

In some cases, package installation may fail with "\ldots{} not found". This likely
means that the package archives point to an old (and non-existant) version of
the package. The package definitions can be updated by running \texttt{M-x
package-refresh-contents}. Packages may also be upgraded all at once with \texttt{M-x
package-upgrade-all}.
\subsection{Initialize}
\label{sec:org2165807}

Initialize the package archive. This makes all previously installed packages available.

\begin{verbatim}
(package-initialize)
\end{verbatim}
\section{Startup}
\label{sec:orgf92848c}

Disable the default Emacs startup screen. Instead, this displays just the opened
file or the \texttt{*Scratch*} buffer if no file has been opened.

\begin{verbatim}
(setq-default inhibit-startup-screen t)
\end{verbatim}
\section{File System}
\label{sec:orgadf2d71}

\subsection{Auto-Revert}
\label{sec:org607b3ad}

Auto revert mode automatically reloads file-based buffer's when their files have
been updated.

\begin{verbatim}
(global-auto-revert-mode t)
\end{verbatim}
\subsection{Backups \& Autosaves}
\label{sec:orgaab9483}

Backup and autosaves may litter the filesystem so we disable
them. This is ok as my disk is reliable, I save often, and use version
control.

\begin{verbatim}
(setq-default auto-save-interval 0
              create-lockfiles   nil
              make-backup-files  nil)
\end{verbatim}
\section{Performance}
\label{sec:orgea78980}

Increase the amount of memory allowed before garbage collection kicks in. Emacs
runs garbage collection when it is over the limit. The limit is equal to the
amount of memory that is \texttt{actively-used} plus \texttt{used-memory *
gc-cons-percentage}. Once the limit is reached, Emacs runs garbage collection
and recomputes the amount of memory that is \texttt{actively-used}.

\begin{verbatim}
(setq-default gc-cons-percentage 1.0)
\end{verbatim}

Run the garbage collector on idle, every 3 seconds. This reduces the chances of
a GC run when interacting with Emacs.

\begin{verbatim}
(run-with-idle-timer 3 t #'garbage-collect)
\end{verbatim}
\section{Appearance \& Feel}
\label{sec:org2e01e3f}

\subsection{Remove Clutter}
\label{sec:orgd7be8ef}

Remove the menu bar and tool bar.

\begin{verbatim}
(menu-bar-mode -1)
(tool-bar-mode -1)
\end{verbatim}

Disable the scroll bar. The functionality is ok sometimes, but it clashes with
the theming.

\begin{verbatim}
(scroll-bar-mode -1)
\end{verbatim}
\subsection{Lines}
\label{sec:orgec02fd1}

Scroll conservatively values above 100 cause Emacs to scroll the
minimum number of lines required to get the cursor in position. The
default value of 0 causes Emacs to recenter the window.

\begin{verbatim}
(setq-default scroll-conservatively 101)
\end{verbatim}

Display line numbers for text buffers. This can be toggled in an individual
buffer with \texttt{M-x display-line-numbers-mode}.

\begin{verbatim}
(global-display-line-numbers-mode t)
\end{verbatim}

Highlight the currently selected line. This can be toggled in an individual
buffer with \texttt{M-x hl-line-mode}.

\begin{verbatim}
(global-hl-line-mode t)
\end{verbatim}
\subsection{Color Scheme}
\label{sec:org070a2a2}

Use the \texttt{doom-dracula} theme from the \href{https://github.com/doomemacs/themes/tree/729ad034631cba41602ad9191275ece472c21941}{Doom Themes} package.

\begin{verbatim}
(add-to-list 'package-selected-packages 'doom-themes)
(load-theme 'doom-dracula t)
\end{verbatim}
\subsection{Modeline}
\label{sec:org3ec4279}

Use \href{https://github.com/seagle0128/doom-modeline/tree/297b57585fe3b3de9e694512170c44c6e104808f}{Doom Modeline} to display a nicer modeline. Mainly, it:

\begin{itemize}
\item Uses more icons.
\item Displays a minimal amount of information while still keeping
important information such as:
\begin{itemize}
\item Syntax errors
\item Version control information
\end{itemize}
\end{itemize}

\begin{verbatim}
(add-to-list 'package-selected-packages 'doom-themes)
(doom-modeline-mode t)
\end{verbatim}
\section{Editor Completions}
\label{sec:org49d8fe6}

Editor completions refers to auto complete done within the editor context, as
opposed to code. Editor completion is used to complete prompts for things such
as selecting a file, buffer, or command.
\subsection{Ivy}
\label{sec:org538c76d}

Editor completions are displayed using the \href{https://github.com/abo-abo/swiper?tab=readme-ov-file\#ivy}{Ivy} package. This provides a huge
improvement over the default built-in Emacs completion.

\begin{verbatim}
(add-to-list 'package-selected-packages 'ivy)
(ivy-mode t)
\end{verbatim}
\subsection{Counsel}
\label{sec:orgb5e94f4}

\href{https://github.com/abo-abo/swiper?tab=readme-ov-file\#counsel}{Counsel} provides functions that wrap ivy completion with some extra
features. For example, \texttt{counsel-M-x} is an \texttt{M-x} replacement that also displays
a keybinding if there is an active keybinding for the particular function.

\begin{verbatim}
(add-to-list 'package-selected-packages 'counsel)
(counsel-mode t)
\end{verbatim}

Enabling \texttt{counsel-mode} makes the \texttt{counsel-mode-map} keymap active. This keymap
defines several rebinds.


However, it does not provide a rebind for \texttt{counsel-switch-buffer}. We make this
our default (interactive) switch buffer command as it allows previewing the
contents of a buffer before switching.

\begin{verbatim}
(define-key counsel-mode-map (kbd "C-x b") #'counsel-switch-buffer)
\end{verbatim}
\section{Formatting}
\label{sec:org0b5f1bd}

\subsection{Tabs}
\label{sec:orgb1dcd13}

Emacs uses a combination of tabs and spaces when auto-indenting. This pleases
neither the spaces nor tabs crowds. Tabs are disabled to prevent the mixed use,
though opinionated languages will still find a way to use their correct
default. For example, Go will still use tabs when indenting.

\begin{verbatim}
(setq-default indent-tabs-mode nil)
\end{verbatim}

Use a default tab width of 4 spaces.

\begin{verbatim}
(setq-default tab-width 4)
\end{verbatim}
\subsection{Line Width}
\label{sec:org011b1db}

Set a target line width of 80. Contents of a "paragraph" may be made to follow
the target line width through \texttt{M-x fill-paragraph} (default keybind \texttt{M-q}) or a
highlighted region with \texttt{M-x fill-region}.

\begin{verbatim}
(setq-default fill-column 80)
\end{verbatim}

Some languages have a different target line length.

\begin{verbatim}
(defun fill-column-100 ()
  (setq-local fill-column 100))

(add-hook 'rust-mode-hook #'fill-column-100)
\end{verbatim}
\section{Languages}
\label{sec:org3405c2b}

\subsection{Rust}
\label{sec:org5dbea34}

\begin{verbatim}
(add-to-list 'package-selected-packages 'rust-mode)
\end{verbatim}
\subsection{Org Mode}
\label{sec:org60fff5f}

Enable syntax highlighting for exported material html. Note that this will use
the currently active theme.

\begin{verbatim}
(add-to-list 'package-selected-packages 'htmlize)
\end{verbatim}
\end{document}
