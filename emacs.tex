% Created 2025-05-31 Sat 10:07
% Intended LaTeX compiler: pdflatex
\documentclass[11pt]{article}
\usepackage[utf8]{inputenc}
\usepackage[T1]{fontenc}
\usepackage{graphicx}
\usepackage{longtable}
\usepackage{wrapfig}
\usepackage{rotating}
\usepackage[normalem]{ulem}
\usepackage{amsmath}
\usepackage{amssymb}
\usepackage{capt-of}
\usepackage{hyperref}
\author{will@wmedrano.dev}
\date{\today}
\title{Emacs Config}
\hypersetup{
 pdfauthor={will@wmedrano.dev},
 pdftitle={Emacs Config},
 pdfkeywords={},
 pdfsubject={},
 pdfcreator={Emacs 30.1 (Org mode 9.7.11)}, 
 pdflang={English}}
\begin{document}

\maketitle
\tableofcontents

\section{Packages}
\label{sec:org5602a42}

\subsection{Melpa}
\label{sec:org33e78ba}

Add Melpa to the package manager. \href{https://melpa.org}{Melpa} contains many popular Emacs packages.

\begin{verbatim}
(require 'package)
(add-to-list 'package-archives '("melpa" . "https://melpa.org/packages/") t)
\end{verbatim}

Counting the default package archives, the following package archives are
available.

\begin{center}
\begin{tabular}{ll}
gnu & \url{https://elpa.gnu.org/packages/}\\
nongnu & \url{https://elpa.nongnu.org/nongnu/}\\
melpa & \url{https://melpa.org/packages/}\\
\end{tabular}
\end{center}

For this Emacs setup, the following packages are required:

\begin{verbatim}
(setq-default package-selected-packages
              '(doom-themes
                doom-modeline
                vertico
                consult
                company
                htmlize
                rg
                magit
                rust-mode
                yaml-mode
                markdown-mode
                ))
\end{verbatim}

Package archives must be manually refreshed or fetched with \texttt{M-x
package-refresh-contents}. All packages added to the \texttt{package-selected-packages}
variable can be installed with \texttt{M-x package-install-selected-packages}. Packages
can also be installed on a one-off basis interactively with \texttt{M-x
package-install}.

In some cases, package installation may fail with "\ldots{} not found". This likely
means that the package archives point to an old (and non-existant) version of
the package. The package definitions can be updated by running \texttt{M-x
package-refresh-contents}. Packages may also be upgraded all at once with \texttt{M-x
package-upgrade-all}.
\subsection{Initialize}
\label{sec:orgb8540e7}

Initialize the package archive. This makes all previously installed packages
available.

\begin{verbatim}
(package-initialize)
\end{verbatim}
\section{Startup}
\label{sec:org14195a4}

Disable the default Emacs startup screen. Instead, this displays just the opened
file or the \texttt{*Scratch*} buffer if no file has been opened.

\begin{verbatim}
(setq-default inhibit-startup-screen t)
\end{verbatim}
\section{File System}
\label{sec:orgabe35a4}

\subsection{Auto-Revert}
\label{sec:orgb91ab99}

Auto revert mode automatically reloads file-based buffer's when their files have
been updated.

\begin{verbatim}
(global-auto-revert-mode t)
\end{verbatim}
\subsection{Backups \& Autosaves}
\label{sec:orge0e4189}

Backup and autosaves may litter the filesystem so we disable
them. This is ok as my disk is reliable, I save often, and use version
control.

\begin{verbatim}
(setq-default auto-save-interval 0
              auto-save-default  nil
              create-lockfiles   nil
              make-backup-files  nil)
\end{verbatim}
\section{Performance}
\label{sec:orgedd83ef}

Increase the amount of memory allowed before garbage collection kicks in. Emacs
runs garbage collection when it is over the limit. The limit is equal to the
amount of memory that is \texttt{actively-used} plus \texttt{used-memory *
gc-cons-percentage}. Once the limit is reached, Emacs runs garbage collection
and recomputes the amount of memory that is \texttt{actively-used}.

\begin{verbatim}
(setq-default gc-cons-percentage 1.0)
\end{verbatim}

Run the garbage collector on idle, every 3 seconds. This reduces the chances of
a GC run when interacting with Emacs.

\begin{verbatim}
(run-with-idle-timer 3 t #'garbage-collect)
\end{verbatim}
\section{Appearance \& Feel}
\label{sec:org0e86dc9}

\subsection{Remove Clutter}
\label{sec:org1435df5}

Remove the menu bar and tool bar.

\begin{verbatim}
(menu-bar-mode -1)
(tool-bar-mode -1)
\end{verbatim}

Disable the scroll bar. The functionality is ok sometimes, but it clashes with
the theming.

\begin{verbatim}
(scroll-bar-mode -1)
\end{verbatim}
\subsection{Lines}
\label{sec:orge12f321}

Scroll conservatively values above 100 cause Emacs to scroll the
minimum number of lines required to get the cursor in position. The
default value of 0 causes Emacs to recenter the window.

\begin{verbatim}
(setq-default scroll-conservatively 101)
\end{verbatim}

Display line numbers for text buffers. This can be toggled in an individual
buffer with \texttt{M-x display-line-numbers-mode}.

\begin{verbatim}
(global-display-line-numbers-mode t)
\end{verbatim}

Highlight the currently selected line. This can be toggled in an individual
buffer with \texttt{M-x hl-line-mode}.

\begin{verbatim}
(global-hl-line-mode t)
\end{verbatim}
\subsection{Color Scheme}
\label{sec:orgdd2c885}

Use the \texttt{doom-dracula} theme from the \href{https://github.com/doomemacs/themes/tree/729ad034631cba41602ad9191275ece472c21941}{Doom Themes} package.

\begin{verbatim}
(load-theme 'doom-dracula t)
\end{verbatim}
\subsection{Modeline}
\label{sec:org796b43d}

Use \href{https://github.com/seagle0128/doom-modeline/tree/297b57585fe3b3de9e694512170c44c6e104808f}{Doom Modeline} to display a nicer modeline. Mainly, it:

\begin{itemize}
\item Uses more icons.
\item Displays a minimal amount of information while still keeping
important information such as:
\begin{itemize}
\item Syntax errors
\item Version control information
\end{itemize}
\end{itemize}

\begin{verbatim}
(doom-modeline-mode t)
\end{verbatim}
\section{Editor Completions}
\label{sec:org99ebd46}

Editor completions refers to auto complete done within the editor context, as
opposed to code. Editor completion is used to complete prompts for things such
as selecting a file, buffer, or command.
\subsection{Vertico}
\label{sec:org0800f06}

Editor completions are displayed using the \href{https://github.com/minad/vertico}{Vertico} package. This provides a huge
improvement over the default built-in Emacs completion.

\begin{verbatim}
(vertico-mode t)
\end{verbatim}
\subsection{Consult}
\label{sec:org6a4a954}

\href{https://github.com/minad/consult}{Consult} provides functions that wrap vertico completion with some extra
features. For example, \texttt{consult-buffer} is an \texttt{switch-to-buffer} replacement
that also displays a keybinding if there is an active keybinding for the
particular function.
\subsection{Buffer Switching}
\label{sec:orgd93ea28}

\begin{verbatim}
(global-set-key (kbd "C-x b") #'consult-buffer)
\end{verbatim}

For some cases, I like to see all the buffers on a buffer instead of the
minibuffer. This can be done \texttt{C-x C-b} which runs \texttt{list-buffers}, but \texttt{ibuffer}
looks a bit nicer.

\begin{verbatim}
(global-set-key (kbd "C-x C-b") #'ibuffer)
\end{verbatim}
\section{Searching}
\label{sec:org28453a2}

\subsection{RG}
\label{sec:org7bac3ba}

The \href{https://github.com/dajva/rg.el}{rg} package can be used to search for regex/literals across a project or
directory. The package can with several general \texttt{rg-} interactive functions. The
most general of which being \texttt{rg-menu}.
\section{Version Control}
\label{sec:org83e8984}

\subsection{Git/Magit}
\label{sec:orgbc6483f}

\href{https://github.com/magit/magit/tree/e3806cbb7dd38ab73624ad48024998705f9d0d20}{Magit} provides an Emacs interface for Git. This involves things such as viewing
diffs, staging, comitting, branching, and many other things. The Magit
documentation provides a lot (too much) information on how to use Magit.

For a quickstart, try running \texttt{M-x magit} to bring up the magit status buffer
and pressing \texttt{?} to see all the commands.
\section{LSP}
\label{sec:org44aec9d}

\subsection{Background}
\label{sec:org513aac3}
The LanguageServerProtocol defines a way for a language server to communicate
programming language specific information for a project to an IDE(Emacs). The
protocol defines things such as syntax checking, autocomplete, and code
formatting.
\subsection{Updating Eglot Package}
\label{sec:org87de168}

Eglot is included in Emacs. However, Eglot can be upgraded to the latest version
with \texttt{M-x eglot-upgrade-eglot}.
\subsection{Obtaining Language Servers}
\label{sec:org255be90}

Eglot is configured to run the most popular language servers by
default. However, they must still be installed on the system. Some popular
language servers include \texttt{rust-analyzer} for Rust.
\subsection{Enabling Eglot}
\label{sec:org17e105c}

Eglot can be manually enabled on a buffer with \texttt{M-x eglot}. To enable it
automatically, you may call \texttt{eglot-ensure} on the buffer automatically through
hooks.

\begin{verbatim}
(add-hook 'rust-mode-hook #'eglot-ensure)
\end{verbatim}
\section{Formatting}
\label{sec:org144ad6c}

\subsection{Tabs}
\label{sec:org9b81f94}

Emacs uses a combination of tabs and spaces when auto-indenting. This pleases
neither the spaces nor tabs crowds. Tabs are disabled to prevent the mixed use,
though opinionated languages will still find a way to use their correct
default. For example, Go will still use tabs when indenting.

\begin{verbatim}
(setq-default indent-tabs-mode nil)
\end{verbatim}

Use a default tab width of 4 spaces.

\begin{verbatim}
(setq-default tab-width 4)
\end{verbatim}
\subsection{Line Width}
\label{sec:org7ce5f89}

Set a target line width of 80. Contents of a "paragraph" may be made to follow
the target line width through \texttt{M-x fill-paragraph} (default keybind \texttt{M-q}) or a
highlighted region with \texttt{M-x fill-region}.

\begin{verbatim}
(setq-default fill-column 80)
\end{verbatim}

Some languages have a different target line length.

\begin{verbatim}
(defun fill-column-100 ()
  (setq-local fill-column 100))

(add-hook 'rust-mode-hook #'fill-column-100)
\end{verbatim}
\subsection{Delete Trailing Whitespace}
\label{sec:org36e1fbb}

Trailing whitespace is usually unintended. These are whitespace characters
hanging at the end of the sentence or newlines/whitespace at the end of the
file. Trailing whitespace can be automatically deleted before save.

\begin{verbatim}
(add-hook 'before-save-hook #'delete-trailing-whitespace)
\end{verbatim}
\subsection{Language Specific Autoformat}
\label{sec:org1dfce63}

Eglot provides 2 functions for formatting.

\begin{itemize}
\item \texttt{eglot-format} - Formats the selected region.
\item \texttt{eglot-format-buffer} - Format the current buffer.
\end{itemize}

However, running these functions interactively is not needed as we can
automatically run \texttt{eglot-format-buffer} before save.

\begin{verbatim}
(defun eglot-maybe-format-buffer ()
  (when (eglot-managed-p) (eglot-format-buffer)))

(add-hook 'before-save-hook #'eglot-maybe-format-buffer)
\end{verbatim}
\section{Code Auto-Complete}
\label{sec:org157304c}

Code auto-complete is handled by the \href{https://company-mode.github.io/}{Company} package. Company is an
autocompletion frontend that comes with many backends. Company comes with a lot
of built-in backends and usually selects the best choice among them for
auto-complete suggestions. One of the more useful backends is the Eglot backend
which is automatically used if the buffer has Eglot mode enabled.

Company mode is usually fine to enable globally. If the buffer doesn't have a
suitable backend, then it does nothing.

\begin{verbatim}
(global-company-mode t)
\end{verbatim}
\section{Syntax Checking}
\label{sec:orgcadd097}

\subsection{On The Fly Syntax Checking}
\label{sec:org12ccfde}

Syntax errors are surfaced by the built-in Flymake package. The Flymake package
provides the frontend and several backends.

The most common backend for Flymake is Eglot. Additionally, Eglot automatically
enables Flymake on Eglot managed buffers so there is no setup on that front.
\subsection{Custom Compile}
\label{sec:orgb261bb4}

\texttt{M-x compile} and \texttt{M-x project-compile} can be used to run a command. \texttt{compile}
runs in the current directory while \texttt{project-compile} runs the command in the
projec's root directory, where a project is often defined as a version
controlled (like Git) repo.

The outputs of the command will be displayed in a buffer named
"\textbf{compilation}". By default, pressing \texttt{n} and \texttt{p} can be used to go through all
detected errors in the buffer. For other keybindings, run \texttt{M-x describe-keymap}
to check \texttt{compilation-mode-map}.
\subsubsection{Error Detection}
\label{sec:orgbb8f9a1}

\texttt{M-x compile} has some built in mechanisms to detect errors. However, some
packages, like \texttt{rust-mode}, add new patterns. For these patterns to be added,
the package has to be loaded. Packages are often lazily loaded or can be
manually loaded with something like \texttt{M-: (require 'rust-mode)}.
\section{Languages}
\label{sec:org6942680}

\subsection{Rust}
\label{sec:org1224eac}

Rust is not built into Emacs so we install the \href{https://github.com/rust-lang/rust-mode/tree/25d91cff281909e9b7cb84e31211c4e7b0480f94}{Rust Mode} package.

Beyond that, there is not much to Rust as most of its functionality comes
throught Eglot + the \texttt{rust-analyzer} LSP.
\subsection{Org Mode}
\label{sec:org9170620}

Enable syntax highlighting for exported material html. Note that this will use
the currently active theme. This requires the \href{https://elpa.nongnu.org/nongnu/htmlize.html}{htmlize} package.
\subsubsection{Indentation}
\label{sec:org1730eba}

The built-in and on by default \texttt{electric-indent-mode} package has buggy behavior
with org lists and buffers so we turn it off.

\begin{verbatim}
(defun electric-indent-mode-local-off ()
  (electric-indent-local-mode -1))

(add-hook 'org-mode-hook #'electric-indent-mode-local-off)
\end{verbatim}
\section{Elisp Concepts}
\label{sec:org786362a}

\subsection{Interactive Commands}
\label{sec:orgaf89e6f}

Interactive functions that can be run "interactively". Here, interactively maens
that they can be run through \texttt{M-x}. Interactive functions are defined by adding
(interactive) in their function definition.

\begin{verbatim}
(defun my-function ()
  "Do a thing."
  (message "Hello World"))

(defun my-interactive-function ()
  "Do a thing."
  (interactive)
  (message "Hello World"))
\end{verbatim}

Interactive functions also have a sophisticated mechanism of querying the user
for standard options and passing them as flags. See the \href{https://www.gnu.org/software/emacs/manual/html\_node/elisp/Using-Interactive.html}{Emacs Documentation} for
"Using interactive" for more details.
\section{{\bfseries\sffamily TODO} Improvements}
\label{sec:orgaeceefe}

\begin{itemize}
\item Remove "Yes" confirmation for changed files.
\item Make checkboxes less ugly.
\item Better PDFs in Emacs.
\end{itemize}
\end{document}
